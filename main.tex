%==
%	DOCUMENT DEFINITION
%==

\documentclass[pdftex,11pt,titlepage,twoside,openright]{report}	


% for suporting multi platform
\usepackage[utf8x]{inputenc} 	

% natbib is what you want for bibliography
\usepackage[square,authoryear]{natbib}


%--
%	DECLARATIONS
%--

\input{./includes/declarations.tex}


%--
%	INDEX AND GLOSSARY
%--


\usepackage[style=long,nonumberlist,toc,xindy,acronym,nomain]{glossaries} % nomain, if you define glossaries in a file, and you use \include{INP-00-glossary}
%\loadglsentries[main]{glossary}
% or using \input:
\include{glossary}
\makeglossaries
\usepackage{makeidx}
\makeindex



%==
%	BEGIN DOCUMENT
%==


\begin{document}

% roman numbering 
\setcounter{page}{1}
\pagenumbering{roman}

% print title
\begin{titlepage}


%TIKZ BACKGROUND
\TITLEBOX

%CONTENT
\begin{center}



\sffamily\textsc{\huge{\textcolor{white}{The University of Texas at Austin}}}\\[4cm]


% Upper part of the page. The '~' is needed because \\
% only works if a paragraph has started.
\includegraphics[width=0.35\textwidth]{media/faculty.png}\makebox[1.5cm]{}\includegraphics[width=0.35\textwidth]{media/unilogo.png}~\\[1.5cm]



% Title
\HRule \\[0.4cm]
{
 \huge \bfseries \sffamily Investigation of Transformer-based Methods for Image Compression, Analysis, and Generation  \\[0.4cm] 
}

\HRule \\[0.4cm] 

\normalfont \LARGE \sffamily Master's Thesis Report \\[1.5cm]
\normalfont \large \sffamily PRE-SUBMISSION DRAFT \\[0.5cm]
\normalfont \large \sffamily Official code repository: \\
\normalfont \large \sffamily \url{https://github.com/kliment-slice/thesis-latex} \\[1.0cm]

% Author and supervisor
\noindent
% \begin{minipage}{0.4\textwidth}
% \begin{flushleft} \large
% \emph{Supervisors:} \\
% Prof. Dr. Alan C. Bovik\\
% Prof. Dr. Rachel A. Ward\\
% \end{flushleft}
% \end{minipage}%
\begin{minipage}{0.4\textwidth}
\begin{center} \large
\emph{Author:}\\
Kliment Minchev\\
\end{center}
\end{minipage}

\vfill


% Bottom of the page
{\large \today}

\vfill



% \vfill
\end{center}
\end{titlepage}



%pagestyle after title
\pagestyle{fancy}

% table of contents,list of figures and list of tables
\setcounter{tocdepth}{1}
% \tableofcontents
% \listoffigures
% \listoftables
{\tableofcontents \let\cleardoublepage\clearpage \listoffigures 
\let\cleardoublepage\clearpage \listoftables \let\cleardoublepage\clearpage}

%\cleardoublepage

% ABSTRACT EXECUTIVE SUMMARY

\begin{abstract}
    This work investigates a novel application of a Vision Transformer to quality 
    analysis of generated images from neural image compression. The Vision Transformer (ViT) 
    is a revolutionary implementation of the Transformer attention mechanism 
    (typically used in language models) to object detection in digital images. 
    Since the ViT architecture is designed to output classification probability distribution, 
    it can be a suitable candidate for quality assessment of generated images based on 
    object-level deviations from the original pre-compression image, rather than measuring 
    per-pixel discrepancies or structural aberrations. Neural image compression and generation is achieved 
    using a Generative Adversarial Network (GAN). This study demonstrates the coveted end-to-end 
    deep learning pipeline for image compression, latent vector representation, regeneration, 
    and quality analysis using state-of-the-art model architectures. Results from this work show 
    that a Vision Transformer is capable of evaluating the quality of a compressed image and compares 
    to established perceived quality metrics such as Structural Similarity (SSIM) and Mean Squared Error (MSE). 
\end{abstract}

\pagestyle{fancy}

\setcounter{page}{1}
\pagenumbering{arabic}


%==
%	CHAPTER 1
%==

\chapter{Introduction to Vision Transformers}

% Chapter overview 
This chapter will present an introduction to Vision Transformers (ViT).

It will cover the motivation as to why ViT, or a future development inspired by it,
will have a profound impact on image compression, analysis, and generation overall.
Evidence will be shown that Vision Transformers, or perhaps an evolved deep learning model with a
similar architecture, i.e. generalizable and highly overparameterized, can be superior in
compressing and evaluating the latent feature space of a digital image compared to known
technologies.

The chapter will then review a brief history or Transformer usage in deep learning.
These generalized architectures are dominating state-of-the-art language models,
as they are extremely efficient in packing information within a one dimensional vector.

This introduction will further describe the principles of operation of a ViT. Then, proceed with
a mathematical formulation. Finally, it will cover currently available implementations
in the form of pre-trained models and conclude with an explanation on the computational and
financial constraints of training such architectures.

\newpage
%--
%	CHAPTER 1
%--

\section{Motivation}

Transformers are presently considered to hold a great promise for the future of Deep Learning
as a step towards Artificial General Intelligence.
Due to their architecture, they are more generalizable, less prone to overfitting, and able 
to learn highly complex representations. The Transformer architecture has already been proven 
to make obsolete Recurrent Neural Networks (RNNs) in natural language models. Furthermore, the 
Vision Transformer (ViT) has outperformed certain Convolutional Neural Networks (CNNs) in image classification tasks. \citep{dosovitskiy2020vit}

Figure 1.1 below shows an increase in the popularity of research related to Vision Transformers.

\begin{figure}[H]
	\begin{center}
	\includegraphics[width=1\textwidth]{media/papersWcode_tfUsage2.png}
	\end{center}
	\caption[Historical Usage of ViT in Image Tasks]{As of 2022, the usage of a Vision Transformer (ViT) in image 
	tasks matches the usage of ResNets and has outnumbered any other popular CNN architecture.
	\citep{PapersOverTime}}
	\end{figure}

Figure 1.1 was produced by PapersWithCode, a popular academic research aggregator. For the past three years, 
ResNets, the most popular architecture in image processing and computer vision, has dominated the proportion
of academic research in object detection. In 2022, Vision Transformer research popularity has reached
that of ResNets and exceeded any other major category.


In the zeitgeist of Vision Transformer research, this thesis will explore a ViT-assisted metric related to 
image compression. This metric can provide additional insights to GAN output quality 
and the latent space (contextual) preservation of a variety of input images.


\NOTE{Thus, this work can be viewed as a stepping stone towards an end-to-end Transformer-based
image compression and regeneration framework.}


\section{Brief History}

\subsection{Attention and Language Models}

"Attention Is All You Need" is a seminal research publication
by a team of Google researchers, which kickstarted the Transformer revolution in Deep Learning in 2017.
It proposes a novel architecture, which models long-range dependencies in
sequential (text) data, by arranging a set of self-attention layers. 

A self-attention layer is what the model uses to focus on different elements of the 
input sequence simultaneously. For example, it can be ysed to compute the 
distance (relationship) between every word in a given sentence. \citep{Attention}


Examples of implementations of text-based Transformers are BERT by Google and GPT-3 by OpenAI.
BERT, among other things, as of 2021 processes and autofills every single English-based 
Google user search query. \citep{bert}
GPT-3, on the other hand, revolutionized text generation in 2020, demonstrating the ability 
to generate extremely cohesive textual output.

Most Transformers are used in language modeling and Natural Language Processing (NLP).
Thus, they are often benchmarked against Recurrent Neural Networks (RNNs, and specifically 
Long Short-Term Memory, LSTM architecture). LSTMs rely on hidden states to pass information 
along sequentially during the encoding and decoding process for each word token. 
However, they typically fall short learning long-range dependencies.


\subsection{Attention in Vision Tasks}

The attention mechanism is capable of focusing on objects found anywhere on an input image.
It operates within a single network layer compared to Convolutional Neural Networks (CNNs),
where the variable size convolution kernels scan across the different layers of the architecture.
\citep{dosovitskiy2020vit}

Tokenization happens at the pixel level, i.e. each pixel attends to each other pixel in the grid. 
This becomes computationally intensive, on the order of $(n^2)^2$, where $n$ denotes width of a square image. 
To resolve this, the input image is broken down into square blocks of equal size, referred to as image patches.
Then, each image patch is unrolled into a one-dimensional sequence $(n x 1)$ and indexed with a positional 
embedding in a table for future reference and retrieval purposes. The embeddings enter the Transformer and 
finally, a feed forward classifier, in the form of a Multilayer Perceptron (MLP) 
makes the classification prediction, yielding a probability distribution.\citep{dosovitskiy2020vit}


Transformer, in a way, is a generalization of a feed forward network, but 
instead of fixed connections weights in an MLP, each connection weight (i.e. attention) 
is computed ad hoc. That makes the Transformer, unlike the MLP, permutation invariant. 
That is, it would not know where certain information is coming from, unless there are additional 
learnable sequential positional embeddings, i.e. index the image patches.


\section{Principles of Operation}

Continuing from the previous section, a good way to think of a ViT is as a 
generalization of an MLP, which itself is a generalization of a CNN. 
The ViT happens to learn very similarly to a CNN, which represents the latent space
as filters carrying principal components.

In principle, CNNs have good inductive priors and can learn any function. 
However, they promote locality, i.e. nearing pixels are probability-wise considered most important.
This may easily not be desired, especially in the key applications of object detection and, in the 
future, image compression.


The encoding process indexes embeddings. For instance, certain key words in a sentence or 
objects in image blocks are mapped in a reference lookup table. 
The Decoder outputs Keys at each step. These vectors represent hidden states,
which are being passed on into each next iteration of the Transformer. 
The last layer, expectedly, uses a Softmax architecture to normalize and map the potential output classes 
to a probability distribution. 


\textbf{Multi-Head Attention}


Sets of parallel attention layers at each token are called multi-head attention 
(to vary what to pay attention to: e.g. at verbs in the sentence or different objects 
in an image). The multi-head attention is composed of Key-Value pairs coming from 
the encoding part of the source sentence or image (i.e. the input embedding) and 
Queries from the output embedding (i.e. encoding part of target sentence or image).
Now that we explained relevant transformer components, we can see how it applies to 
2D signals, i.e. image matrices for classification purposes in image recognition.


\section{Mathematical Formulation}

So in its full formulation, Attention is a function of queries, keys and values 
vectors labeled capital $(Q,K,V)$. 
It equals the dot product $(QK^T)$ of keys and queries respectively, softmaxed over 
the square root of dimensions and multiplied by Values.
So:
Values - are what is most interesting in the source (sentence or image), 
e.g. attributes or features (like keyword adjectives or perhaps structural 
features in an image).
Keys - index (or address) those values (name, type, weight). Each key has an 
associated value. Queries - are built by the encoder of the target sentece or 
image and prompt the network to find information. 

\begin{center}
	$ Attention(Q, K, V) = softmax(\dfrac{QK^T}{\sqrt{d_k}} V ) $
	\end{center}

The overall dynamic: a Query is pegged against a Key to locate a certain Value.	

The softmax is basically a normalized exponential function: sequence of Variables is 
mapped into exponentials and divided by the sum of all the exponentials. Thus, the 
large numbers become almost ones and small numbers near zeros, like the maximum 
function, but this one is differentiable. 

\begin{center}
	$ \displaystyle\sigma(Z_i) = \dfrac{e^{z_j}}{\sum_{j=1}^K e^{z_j}}$ for $i=1,...,K$ $and z=(z_1,...,z_K) \in \mathbb{R}^K $
	\end{center}

So a softmax of an inner product of each key with query vector normalizes to a 
probability distribution over all Values (very similar to using a softmax in the last 
layer of a NN over all the labels to yield the top classification pick). 


To understand vector proximity between embeddings, e.g. similarity in objects for images, or words in sentences.

\begin{figure}[H]
	\begin{center}
	\includegraphics[width=0.5\textwidth]{media/dot_real.png}
	\end{center}
	\caption[Vector Representation]{Vector representation on a unit circle.}
	\end{figure}

The dot product of Keys and Queries yields an angle between both vectors (e.g. $u_i$ and $u_j$ in Figure 1.2 above) to measure 
how similarly aligned they are. In high dimension, most vectors would be orthonormal 
and $cos(90)=0$. But if Key and Query align, they'd have a large dot product. 
Each key in space has an associated value (the pair). The Query vector is computed with 
each key and softmaxed to select one Key with the highest dot product. With softmax, 
a certain Key will stand out (in magnitude) vs the rest.

\begin{figure}[H]
	\begin{center}
	\includegraphics[width=0.3\textwidth]{media/dot2_real.png}
	\end{center}
	\caption[Key/Query Vector Proximity]{Vector proximity shows the closest Key vector to a given Query vector.}
	\end{figure}

So the formulation of each proximity is the dot product of vectors K with Q: $ \displaystyle sum(\langle K_z \vert Q \rangle) $

%\newpage

\section{Implementations}
\subsection{Open Source Pre-trained Models}

Open Source
On GitHub.
Original one by Google in tensorflow.
Hugging Face in PyTorch.
For the purpose of this thesis, we will use a PyTorch implementation trained on ImageNet-21k
and fine tuned on ImageNet-1k.

\subsection{Closed Source}

OpenAI (DALL.E 2) released in April 2022, trained on 250M image-text pairs 
to be able to generate images from textual description. \citep{Dalle2}
Not much information, not open source like google's ViT or BERT.

\section{Computational Constraints}

Time, memory, cost

Text transformers perform extremely well on orders of magnitude smaller size training examples.
GPT-3 trained on 45TB of text data (Wikipedia included), has 175B parameters and 96 attention layers. \citep{GPT3}
\$4.6M using a Nvidia advanced datacenter GPU grade cloud cluster.

Several orders of magnitude more for image data.

Need a ViT on all internet, to cost \$100M in training alone. 
To train a ViT on the whole TACC Frontera at 20k teraflops (top10) or 
Stampede at 10k tflops (top25), it would take respectively about a minute and 2 minutes.
tpu v3 is 420 teraflops * 2500 = 1M Tflops











%==
%	CHAPTER 2
%==
\cleardoublepage
\chapter{Literature Review: ViT and Neural Image Compression}

Now that we explained relevant transformer components, we can see how it applies to 2D signals, i.e. image matrices for classification purposes in image recognition.
The Vision Transformer or ViT, was very recently published for ICLR 2021 by a google team for a ImageNet trained transformer.
Tokenization happens at pixel level, so each pixel would have to attend to  each other pixel in the grid, which becomes too heavy to compute, on the order of
To resolve that, the image is broken down into blocks of equal size, a 16x16 subset of the image called image patches. Then, unroll each image patch into a sequence (256x1), and index it with a positional embedding in a table. All of that is then fed into a standard Transformer, like from Attention is all you need. Finally, a feed forward classifier (MLP) makes the classification prediction, voila image recognition.
The total number of parameters is on the order of 100M.

\newpage
\section{"An Image is Worth 16x16 Words"}

\subsection{Transformers for Image Recognition at Scale}

Now that we explained relevant transformer components, we can see how it applies to 2D signals, i.e. image matrices for classification purposes in image recognition.
The Vision Transformer or ViT, was very recently published for ICLR 2021 by a google team for a ImageNet trained transformer.
Tokenization happens at pixel level, so each pixel would have to attend to  each other pixel in the grid, which becomes too heavy to compute, on the order of $(250^2)^2$. 
To resolve that, the image is broken down into blocks of equal size, a 16x16 subset of the image called image patches. Then, unroll each image patch into a sequence (256x1), and index it with a positional embedding in a table. All of that is then fed into a standard Transformer, like from Attention is all you need. Finally, a feed forward classifier (MLP) makes the classification prediction, voila image recognition.
The total number of parameters is on the order of 100M.

"image is worth 16x16 words" completely discards the notion of convolutions. Compared to a convolutional counterpart (the ResNet), the ViT cost 75% less to train and beats accuracy on ImageNet by 1%. ViT uses approximately 2 − 4× less compute to attain the same performance (averaged over 5 datasets). Unlike variable size convolution kernels accross layers, the block size in the image transformer is able to pay attention within a single layer to anywhere on the image. 
CNNs have good inductive priors and can learn any function. However, this promotes locality (i.e. nearing pixels are probably most important).
One way to think of a ViT is a generalization of an MLP which itself is a generalization of a CNN. The ViT also learns very similarly to a CNN (e.g. filters with principal components).
Transformer, in a way, is a generalization of a feed forward network, but instead of fixed connections weights in an MLP, each connection weight (i.e. attention) is computed on the fly. That makes the Transformer, unlike the MLP, permutation invariant. In that it wouldn't know where information is coming from unless there are additional learnable sequential positional embeddings, i.e. number down the image patches.

% \begin{figure}[H]
% \begin{algorithm}[H]
% \KwData{A as Array to sort,}
% \KwResult{A as sorted Array} 

% int n $\leftarrow$ A.size \tcp*[l]{cache the initial size of A}


% \Repeat{n>1}{
% 	int newn $\leftarrow$ 1
% 	\For{int i=0; i<n-1; i++}
% 	{
% 		\If{A[i] > A[i+1]}
% 		{
% 			A.swap(i, +1);
% 		}
% 	}
% 	n  $\leftarrow$ newn
% }

% return A
% \caption{bubbleSort(Array A)}
% \end{algorithm}
%\caption[TBC]{Figure caption.v.}
% \end{figure}


\newpage
\section{"Towards End-to-End Image Compression with Transformers"}


% \begin{table}[H]
% \begin{center}
% \begin{tabular}{|c|c|c|c|c|c|c|}
% \hline
% result XY & \multicolumn{6}{c|}{Rater A}\\\hline
%  \multirow{6}{*}{Rater B}& & \textbf{9} & \textbf{0} & \textbf{1} & \textbf{2} & \textbf{c(a) }\\\cline{2-7}
%  & \textbf{9}& 27	&0&	0	&0 & 0.5510204082\\\cline{2-7}
% &\textbf{0}	&0	&6	&0	&0&0.1224489796\\\cline{2-7}
% &\textbf{1}	&0	&1	&7	&0&0.1632653061\\\cline{2-7}
% &\textbf{2}	&0	&1	&7	&0&0.1632653061\\\cline{2-7}
% &\textbf{c(r)} & 0.5510204082 & 0.1632653061 & 0.2857142857 & 0& n=49\\\hline

% \end{tabular}
% \caption[Example data for mathematical equations]{A set of example data for further use in a mathematical equation.  }
% \end{center}
% \end{table}

\begin{center}
$Pr(e) = \displaystyle\sum_{i=1}^{m} c(r)_i \times c(a)_i $
\end{center}


\section{"TransGAN"}
\subsection{"Two Pure Transformers Can Make One Strong GAN"}

Sometimes you want to include text which include characters that could trigger commands. At this point it useful to wrap them into the \NOTE{verbatim} environment. The text is uninterpreted and listet as is.

You can also include existing code and highlight it's syntax, using the \NOTE{lstlisting} package. Look at the declarations.tex file for listings settings. The following example highlights xml syntax by given keywords.


\section{"First Principles of Deep Learning and Compression"}

Deterministic, Probabilistic
GANs
JPEG/MPEG

\section{Commonly Used Metrics for Image Quality}

SSIM, MSE, PSNR, BRISQUE

%==
%	CHAPTER 3
%==
\cleardoublepage
\chapter{ViT Assessment of Neural Image Compression (Theory and Practice)}

Now that we explained relevant transformer components, we can see how it applies to 2D signals, i.e. image matrices for classification purposes in image recognition.
The Vision Transformer or ViT, was very recently published for ICLR 2021 by a google team for a ImageNet trained transformer.
Tokenization happens at pixel level, so each pixel would have to attend to  each other pixel in the grid, which becomes too heavy to compute, on the order of. 
To resolve that, the image is broken down into blocks of equal size, a 16x16 subset of the image called image patches. Then, unroll each image patch into a sequence (256x1), and index it with a positional embedding in a table. All of that is then fed into a standard Transformer, like from Attention is all you need. Finally, a feed forward classifier (MLP) makes the classification prediction, voila image recognition.
The total number of parameters is on the order of 100M.

\newpage
\section{Generative Image Compression and Generation Overview}

(vineeth)

\subsection{Architecture}

% \begin{figure}[H]
% \begin{algorithm}[H]
% \KwData{A as Array to sort,}
% \KwResult{A as sorted Array} 

% int n $\leftarrow$ A.size \tcp*[l]{cache the initial size of A}


% \Repeat{n>1}{
% 	int newn $\leftarrow$ 1
% 	\For{int i=0; i<n-1; i++}
% 	{
% 		\If{A[i] > A[i+1]}
% 		{
% 			A.swap(i, +1);
% 		}
% 	}
% 	n  $\leftarrow$ newn
% }

% return A
% \caption{Neural Image Compression}
% \end{algorithm}
% \caption[TBC]{Figure caption caption}
% \end{figure}

\subsection{Latent space vector representation after compression}

% \newpage
\section{Outputs and Results}

Before and after, pandas table
score out of 100

graphs, tables

% \begin{table}[H]
% \begin{center}
% \begin{tabular}{|c|c|c|c|c|c|c|}
% \hline
% result XY & \multicolumn{6}{c|}{Rater A}\\\hline
%  \multirow{6}{*}{Rater B}& & \textbf{9} & \textbf{0} & \textbf{1} & \textbf{2} & \textbf{c(a) }\\\cline{2-7}
%  & \textbf{9}& 27	&0&	0	&0 & 0.5510204082\\\cline{2-7}
% &\textbf{0}	&0	&6	&0	&0&0.1224489796\\\cline{2-7}
% &\textbf{1}	&0	&1	&7	&0&0.1632653061\\\cline{2-7}
% &\textbf{2}	&0	&1	&7	&0&0.1632653061\\\cline{2-7}
% &\textbf{c(r)} & 0.5510204082 & 0.1632653061 & 0.2857142857 & 0& n=49\\\hline

% \end{tabular}
% \caption[Example data for mathematical equations]{A set of example data for further use in a mathematical equation.  }
% \end{center}
% \end{table}


\begin{center}
$Pr(e) = \displaystyle\sum_{i=1}^{m} c(r)_i \times c(a)_i $
\end{center}


\section{Visual Inspection}

(natural performance asymptote)

\section{GAN-Related Quantitative Metrics}

(FID score, inception score)
pytorch-ignite

\section{Established IQA Metrics}
SSIM, MSE, PSNR, BRISQUE (and definitions of each)

\section{Optimization Techniques}

(reducing learning\_rate as the model trains)


%==
%	CHAPTER 4
%==
\cleardoublepage
\chapter{Discussion}

Now that we explained relevant transformer components, we can see how it applies to 2D signals, i.e. image matrices for classification purposes in image recognition.
The Vision Transformer or ViT, was very recently published for ICLR 2021 by a google team for a ImageNet trained transformer.
Tokenization happens at pixel level, so each pixel would have to attend to  each other pixel in the grid, which becomes too heavy to compute, on the order of. 
To resolve that, the image is broken down into blocks of equal size, a 16x16 subset of the image called image patches. Then, unroll each image patch into a sequence (256x1), and index it with a positional embedding in a table. All of that is then fed into a standard Transformer, like from Attention is all you need. Finally, a feed forward classifier (MLP) makes the classification prediction, voila image recognition.
The total number of parameters is on the order of 100M.

\newpage
\section{Discussion of Results}

The results summarized in Section 3.5 demonstrate that a ViT-Score is capable of:

\begin{itemize}
    \item providing higher abstraction (object-level) insights
    \item comparing similar images for aberrations and distortions
	\item assessing the quality of a GAN generated image 
    \item complementing existing metrics as a useful measure of image quality
    \item providing a novel approach to Deep Learning-based compression evaluation
\end{itemize} 

"Tower" was the most incomprehesible generated image. The GAN failed to reconstruct a recognizable
object, which resulted in the lowest ViT-Score among all four input images. 
This score is a testament for the valuable insights the ViT-Score contributes.


"Logo" ViT-Score could be higher, as are other metrics. The ViT object detector potentially struggled with 
finding the exact labels that would match such a unique symbollic graphic.
Furthermore, the original input can be a transparent (backgroundless) vector graphic. 
If the backgrounds are to be ignored, perhaps all scores would increase.


Some of the metrics such as PSNR, FID, and BRISQUE seem to lack significant contributions in their assessment. They have the tendency 
to be unable to capture synthetically induced aberrations, which are obvious to the human eye.
BRISQUE fairly consistently ranks the images in their photorealism. "Logo" is indeed a non-realistic synthetic symbol.
The reconstruction of "Bevo" reminds the reader of an oil painting, yet BRISQUE ranks it most realistic with a low Delta.
BRISQUE Deltas (from Table 3.3) can be insightful as well, since the smallest changes in score determine a level of 
consistency in evaluation.

Furthermore, the generated "Kliment" and "Tower" seem to have a similar PSNR, yet "Kliment" is a clearly defined object, 
whereas "Tower" is the least recognizable one.


In fact, in the case of "Kliment" and "Tower", the \NOTE{ViT-Score proves to be more insightful than an established metric such as PSNR.}


Since features are typically found in the deeper layers of the GAN network, it is substantiated that a lot of structural 
information can be compressed into the 1-dimensional latent vector.
The latent space is hard to decipher, but a Transformer is efficient in packing and unpacking information from it.
The Compression Ratios associated with a fixed size latent vector can be useful in low bandwidth transmission scenarios.

\section{Improvements}

\subsection{ViT-Score}

The ViT-Score (defined in Section 3.3.1) could be further improved or experimented with.
For the purpose of this project, the ViT-Score was based on how many of the top-100 labels match
between the input and generated images. Experiments with the top-$k$ value could yield a more
optimal $k$, since 100 was chosen arbitrarily.

The ViT-Score can also take into account probability of each label found (included in the script output, shown in Figure 3.1).
However, while the label probability is stable when working with corrupted images, it is rather 
unstable when working with generated images. Nonetheless, much like the SSIM (defined in Section 2.5),
an elaborate regularized equation could include the output probabilities per detected label.
In turn, this could improve the ViT-Score. 

Intuitively, the ViT and ViT-score can also be used as a Loss function in the model architecture.


\subsection{Compression Engine}

The pre-trained "PGAN" used in the compression engine needs to be trained on more data inspired by human perception.
An important note is that the Fashion-Gen dataset contains 293,008 high resolution images of size 1360x1360. 
The DTD texture dataset, includes 5640 images, with 120 images in 47 categories of varied resolution between 300x300 and 640x640. 
There is an obvious dataset imbalance, as the pre-trained model used has had more exposure to certain classes of images than other. 
\citep{PytorchGANZoo}


Thus, it can be concluded that the natural performance limit to the capacity of this pre-trained model comes from its training sets.
This applies to both the pre-trained "PGAN" and the pre-trained ViT. 
In a more capable iteration of this project, a GAN and ViT trained on all internet images, or a sizable, diverse, and 
representative subset, is necessary. 


Evaluating output quality from Generative Adversarial Networks (GANs) is 
still a developing field, which uses non-Deep Learning-adapted assessment methods. 
It is expected that new and more capable methods of evaluating generated output will emerge.
Furthermore, using these future GAN metrics as Loss functions inside the compression engine could 
significantly improve the GAN generated output.


Finally, unique positional encoding in the ViT can be achieved using trigonometric representation (e.g. periods of a sinusoid).
This could be extremely useful when scaling the approach to elaborate high-resolution input images with lots of objects. 
For example, the delta between the original and generated images could be used to find and fine tune 
discrepancies between rows of pixels. Thus, as the exact location of each object (token) is known, deviations may 
be used to penalize the output and steer the Generator into improving its output. \citep{dosovitskiy2020vit}


\section{Optimization} 

Some of the optimization techniques used in this work include:
\begin{itemize}
    \item Learning rate reduction on plateau, hyperparameters for patience, threshold, and eps
    \item SGD, hyperparameters for learning rate and momentum 
	\item Varying input images 
\end{itemize} 

One of the most valuable techniques, which aided the GAN in improving its output quality, was
reducing the learning rate as the model trains. Stochastic Gradient Descent (SGD) was used as an optimizer for the GAN.
Perhaps substituting SGD with Adam could yield better results on certain image types, though probably
not on average.


A major opportunity to optimize the compression engine would be to experiment with Loss functions.
Mean Squared Error (MSE) was used throughout this study, however, SSIM or potentially a GAN specific evaluator such as FID
could achieve better results (refer to Section 2.5 for definitions).
Additionally, experimenting with input image types to cater to what the generative model is best trained on
could yield much better results as well.
Other options to experiment with include introducing regularization during training such as residual dropout and label smoothing.


When analyzing the latent space vector, it is possible to engage an adapted Transformer model. This overparameterized model
can create maps from the latent space to the spatial representation of the image. 
This generalizable approach can steer the GAN to train faster, compress better, and output higher quality images.
Finally, Transfer learning or combinations of deterministic and probabilistic methods can combat slow training times 
(as mentioned in Section 2.4).

%\newpage

\section{Present and Future of Image Transformers}

\subsection{Status Quo}
According to Figure 1.1 and Section 1.1, there is a clearly defined increase in interest in using Vision Transformer-based 
models for applications in image processing and computer vision.
Furthermore, there has been a significant increase in relevant recent publications related to the topic of this thesis.
A staggering thirteen (13) publications in 2022 thus far support the vision of tying 
Vision Transformers (ViT) to image compression. \citep{arxiv13}


A clear statement must be made that the present day is still an early stage in the Deep Learning and Artificial Intelligence 
evolutionary process. Many of the models have yet to be trained on sizable amounts of image data.
Hence, the current models are still suffering from underperformance in tasks that are easy to achieve by human perception.

Finally, the existing approaches to image compression, analysis, and generation may not reflect the most efficient ones, but 
rather the most scalable and universally accepted. \citep{Principles}

\subsection{Future Developments}

The future of pre-trained GAN and ViT models will include extremely wide and overparameterized feature sets.
Training will be done on sizable sets of images from the internet to mimic the development of 
the text-based language model Transformers such as GPT-3 (see Section 1.6). 
An extremely large dataset (on the order of petabytes of data) featuring diverse and 
representative images will be compiled and used for training by a major technology company or AI initiative.

\NOTE{A coveted and highly desirable Deep Learning-based image compression technique will emerge, which will succeed 
the use of JPEG and other classical compression techniques.} 


For the purpose of this thesis project, a Convolutional Neural Network (CNN) based Generative Adversarial Network (GAN)
was used as a placeholder for a more generalizable alternative. Perhaps, it could also be Transformer or Vision Transformer-based.
In fact, a publication in Computer Vision and Pattern Recognition (CVPR) 2021 dubbed "TransGAN: Two Pure Transformers Can Make One Strong GAN"
was proposed by a team from The University of Texas at Austin. \citep{jiang2021transgan}

It demonstrates that the Generator and Discriminator in a GAN could be replaced with Transformers free of 
convolutions. These TransGANs were able to match or exceed the performance of similar CNN-based GANs. \citep{jiang2021transgan}

In the future, GANs will improve immensely, as the approach proposed by Ian Goodfellow in 2014 is still only in its 
first generation as a highly promising generative technique.


\NOTE{To conclude the discussion on the future of neural image compression, it can be said that all developments will 
organically extend into neural video compression.}



%==
%	CHAPTER 5
%==
\cleardoublepage
\chapter{Summary}

Now that we explained relevant transformer components, we can see how it applies to 2D signals, i.e. image matrices for classification purposes in image recognition.
The Vision Transformer or ViT, was very recently published for ICLR 2021 by a google team for a ImageNet trained transformer.
Tokenization happens at pixel level, so each pixel would have to attend to  each other pixel in the grid, which becomes too heavy to compute, on the order of. 
To resolve that, the image is broken down into blocks of equal size, a 16x16 subset of the image called image patches. Then, unroll each image patch into a sequence (256x1), and index it with a positional embedding in a table. All of that is then fed into a standard Transformer, like from Attention is all you need. Finally, a feed forward classifier (MLP) makes the classification prediction, voila image recognition.
The total number of parameters is on the order of 100M.

\newpage
\section{Key Contributions}

The main merit of this thesis project is introducing the ViT-Score (see Section 3.3).
It is a Vision Transformer-assisted metric for evaluating the performance of neural image compression. 
It computes a score to compare the reconstructed output from a Generative Adversarial Network (GAN) to the original 
input image.


The ViT-Score conclusively contributes to evaluating image quality by quantifying human perception of recognizable objects in the generated image. 
The metric can also provide additional insights to understanding the latent space (contextual) preservation of an input image.
Finally, the ViT-Score can most likely verify that a GAN has generated a comprehensible image.

This work can also be viewed as contributing towards:

\begin{itemize}
    \item an end-to-end Deep Learning-based approach to image compression and reconstruction 
    \item abstracted evaluation of GAN generated images 
	\item promoting generalizable and overparameterized models in Deep Learning
	\item experimentation with Transformers while still a nascent technology
	\item expansion of human consciousness through investigation of evolving techniques in Artifical Intelligence (AI) 
\end{itemize} 


\section{Summary}

Throughout this report, the reader was presented with all relevant background knowledge necessary 
to grasp the key contributions listed. 


A Vision Transformer (ViT) was used to evaluate the capacity of 
a GAN to compress and generate an image of choice based on object-level similarities with the original input image.


The new metric, referred to as a ViT-Score, was able to capture and assess the quality of the output images and provide 
valuable insights. The ViT-Score performed well, comparing in capacity to established image quality metrics such as
SSIM, MSE, and PSNR. 


\section{Takeaways}

The future of image compression technology will be based on a Deep Learning methodology.
Due to their generalizability, excellent performance in 1-dimensional data (text),
and proven ability to scale to 2-dimensions, Transformers are an excellent choice of 
architecture to use in image compression and reconstruction.


A Vision Transformer (ViT)-Assisted metric related to image compression can provide 
additional insights to understanding latent feature space and its preservation through lossy compression.


Such a metric could be used as a Loss function, embedded within the architecture.
It could be useful as a standalone evaluation metric.


It may cost on the order of \$100M and several years to develop, but an end-to-end deep learning-based approach
to image compression will be achieved.


Finally, such image compression technology can be extended to video and video compression in further developments.


\section{Acknowledgments}

The author would like to express gratitude towards several individuals and organizations
from The University of Texas at Austin campus.
The major inspiration for this project was gathered from two courses taught by the
reviewers of this thesis.


EE 371Q, Digital Image Processing taught by Professor Dr. Alan C. Bovik was the class
where the author learned about 
Image Compression, Image Quality Assessment, and completed a term
project on Generative Adversarial Networks (GANs).


CSE 382, Foundations of Machine Learning taught by Professor Dr. Rachel A. Ward 
was the class where the author learned key concepts used throughout this thesis
and completed a term project on Vision Transformers (ViT).


Further acknowledgments are made to the Laboratory for Image and Video Engineering (LIVE) at the 
University of Texas at Austin for providing a source for project inspiration and insights.

Finally, The Texas Advanced Computing Center (TACC) provided free access to advanced 
High-Performance Computing (HPC) resources, which were used throughout the experimentation process in this thesis.

\section{Closing Remarks}

This thesis is written as a graduation requirement for the degree of Master of Science in
Computational Science, Engineering, and Mathematics awarded by the Oden Institute at 
The University of Texas at Austin.

\vspace{5mm}

\NOTE{All code has been made available as open source to the general public in the form of a GitHub repository.}

%==
%	BIBLIOGRAPHY
%==

\cleardoublepage
\phantomsection %hyperref package support
\addcontentsline{toc}{chapter}{Bibliography} % add entry to table of contents
\pagestyle{plain}


%{\textbf{\LARGE{Bibliography}}}\\	%headline
%\nocitep{*} % Show all Bib-entries (DEBUG)

\bibliographystyle{abbrvnat}
% \bibliography{bib/algorithm.bib,bib/resDes.bib} % for a better structure you can split your bib items into seperate files
\bibliography{bib/resDes.bib}

\cleardoublepage
\phantomsection %hyperref package support
\addcontentsline{toc}{chapter}{Appendix} % add entry to table of contents
\pagestyle{plain}
\subsection{Technologies used}

Python, PyTorch
MATLAB for BRISQUE
LaTeX to generate this PDF

\subsection{GPU, Local machine}

NVIDIA GTX 1650Ti
CUDA 11

Project dependencies (requirements.txt)

\scriptsize
\begin{verbatim}
    kiwisolver==1.3.1
    matplotlib==3.2.0
    matplotlib-inline==0.1.3
    numpy==1.22.3
    opencv-python==4.4.0.46
    packaging==21.3
    pandas==1.4.2
    pickleshare==0.7.5
    Pillow==8.0.1
    pytorch-pretrained-vit==0.0.7
    pywin32==303
    pyzmq==22.3.0
    regex==2020.11.13
    scikit-image==0.18.1
    scipy==1.5.4
    torch==1.7.1+cu110
    torchvision==0.8.2+cu110
\end{verbatim}
\normalsize	

\subsection{TACC, Stampede2, job submission process}

TACC Job submissions

\scriptsize
\begin{verbatim}
    #!/bin/bash

    #SBATCH -J run_model         # Job name
    #SBATCH -o logs/job.%j.out   # Name of stdout output file (%j expands to jobId)
    #SBATCH -e logs/job.%j.err   # error file
    #SBATCH -p gtx               # Queue name
    #SBATCH -N 1                 # Total number of nodes requested (16 cores/node)
    #SBATCH -n 1                 # Total number of tasks requested
    #SBATCH -t 24:00:00          # Run time (hh:mm:ss) - 24 hours
    #SBATCH -A Automatic-Assessment
    
    module load python3
    module load cuda/10.0
    module load cudnn/7.6.2
    
    cd /work/29369/kliment/
    date
    
    model_path="/model/model.1.pkl"
    
    python3 main.py --data_path ./data/ 
    
    date
    
\end{verbatim}
\normalsize	

TACC srun/idev


\small
\begin{lstlisting}[keywordstyle=\color{blue},language=Python]
    cd $WORK2
    idev -m 30
    module load python3
    
    squeue
    python3 transformer.py --data_path 
\end{lstlisting}
\normalsize	

\end{document}