%============================================================================%
%
%
%	DOCUMENT DEFINITION
%
%
%============================================================================%

% we use report class for thesis design
% 11pt font is way bettern to read than 12pt
% we want to make a title page
% two side allows us to make the page layout align by even and odd
% openright tells the compiler that the first page is a right page
\documentclass[pdftex,11pt,titlepage,twoside,openright]{report}	


%----------------------------------------------------------------------------------------
%	ENCODING
%----------------------------------------------------------------------------------------

% for suporting multi platform
\usepackage[utf8x]{inputenc} 	

% natbib is what you want for bibliography
\usepackage[square,authoryear]{natbib}


%----------------------------------------------------------------------------------------
%	ALL DECLARATIONS
%----------------------------------------------------------------------------------------

% here are all our declarations 
\input{./includes/declarations.tex}


%----------------------------------------------------------------------------------------
%	MAKE INDEX AND GLOSSARY
%----------------------------------------------------------------------------------------


\usepackage[style=long,nonumberlist,toc,xindy,acronym,nomain]{glossaries} % nomain, if you define glossaries in a file, and you use \include{INP-00-glossary}
%\loadglsentries[main]{glossary}
% or using \input:
\include{glossary}
\makeglossaries
\usepackage{makeidx}
\makeindex



%============================================================================%
%
%	BEGIN DOCUMENT
%
%============================================================================%


\begin{document}

% before the chapters start, we use roman numbering on the pages
\setcounter{page}{1}
\pagenumbering{roman}

% print title
\begin{titlepage}


%TIKZ BACKGROUND
\TITLEBOX

%CONTENT
\begin{center}



\sffamily\textsc{\huge{\textcolor{white}{The University of Texas at Austin}}}\\[4cm]


% Upper part of the page. The '~' is needed because \\
% only works if a paragraph has started.
\includegraphics[width=0.35\textwidth]{media/faculty.png}\makebox[1.5cm]{}\includegraphics[width=0.35\textwidth]{media/unilogo.png}~\\[1.5cm]



% Title
\HRule \\[0.4cm]
{
 \huge \bfseries \sffamily Investigation of Transformer-based Methods for Image Compression, Analysis, and Generation  \\[0.4cm] 
}

\HRule \\[0.4cm] 

\normalfont \LARGE \sffamily Master's Thesis Report \\[1.5cm]
\normalfont \large \sffamily PRE-SUBMISSION DRAFT \\[0.5cm]
\normalfont \large \sffamily Official code repository: \\
\normalfont \large \sffamily \url{https://github.com/kliment-slice/thesis-latex} \\[1.0cm]

% Author and supervisor
\noindent
% \begin{minipage}{0.4\textwidth}
% \begin{flushleft} \large
% \emph{Supervisors:} \\
% Prof. Dr. Alan C. Bovik\\
% Prof. Dr. Rachel A. Ward\\
% \end{flushleft}
% \end{minipage}%
\begin{minipage}{0.4\textwidth}
\begin{center} \large
\emph{Author:}\\
Kliment Minchev\\
\end{center}
\end{minipage}

\vfill


% Bottom of the page
{\large \today}

\vfill



% \vfill
\end{center}
\end{titlepage}



%pagestyle after title
\pagestyle{fancy}

% let's print the table of content, 
% the list of figures and the list of tables
\setcounter{tocdepth}{1}
% \tableofcontents
% \listoffigures
% \listoftables
{\tableofcontents \let\cleardoublepage\clearpage \listoffigures 
\let\cleardoublepage\clearpage \listoftables \let\cleardoublepage\clearpage}

%\cleardoublepage

% EXECUTIVE SUMMARY
% here we describe roughly what this work is about

\begin{abstract}

    Vision Transformers (ViT) \citep{Bai2022AAAI} and TransGANs[2] are state-of-the-art implementations of the 
    transformer attention mechanism used in generative image tasks. The transformer architecture 
    has outperformed convolutional neural networks (CNNs) in image recognition and object detection 
    due to its ability to generalize well. The attention mechanism is capable of focusing on objects 
    anywhere on the image within a single network layer (versus the variable size convolution kernels 
    across layers). The standard two CNNs combined in a generative adversarial network (GAN), can be 
    replaced with vision transformers to create a TransGAN[2]. Such a model is shown to generalize well 
    and may be capable of handling image and video compression tasks. This thesis project investigates 
    the capacity to which a transformer and potentially a TransGAN is capable of achieving high fidelity 
    video compression in order to mimic the performance of common video codecs. 
\end{abstract}

% Back to arabic numbering

\pagestyle{fancy}

\setcounter{page}{1}
\pagenumbering{arabic}


%============================================================================%
%
%	CHAPTER 1
%
%============================================================================%

\chapter{General Tools and Techniques for Your Master Thesis}

% before the chapter starts, we can write some overview 
% this is good to summarize quickly what this chapter covers
At this point it may be good for the reader to get an overview about this chapter. What does it cover? What is it's purpose in the bigger picture of your thesis? It is like a micro-abstract and it helps your audience to keep on track. This is especially useful, when your thesis is an interdisciplinary work.



\newpage
%--
%	CHAPTER 1
%--

\section{Motivation}

Transformers are presently considered to hold a great promise for the future of Deep Learning
as a step towards Artificial General Intelligence.
Due to their architecture, they are more generalizable, less prone to overfitting, and able 
to learn highly complex representations. The Transformer architecture has already been proven 
to make obsolete Recurrent Neural Networks (RNNs) in natural language models. Furthermore, the 
Vision Transformer (ViT) has outperformed certain Convolutional Neural Networks (CNNs) in image classification tasks. \citep{dosovitskiy2020vit}

Figure 1.1 below shows an increase in the popularity of research related to Vision Transformers.

\begin{figure}[H]
	\begin{center}
	\includegraphics[width=1\textwidth]{media/papersWcode_tfUsage2.png}
	\end{center}
	\caption[Historical Usage of ViT in Image Tasks]{As of 2022, the usage of a Vision Transformer (ViT) in image 
	tasks matches the usage of ResNets and has outnumbered any other popular CNN architecture.
	\citep{PapersOverTime}}
	\end{figure}

Figure 1.1 was produced by PapersWithCode, a popular academic research aggregator. For the past three years, 
ResNets, the most popular architecture in image processing and computer vision, has dominated the proportion
of academic research in object detection. In 2022, Vision Transformer research popularity has reached
that of ResNets and exceeded any other major category.


In the zeitgeist of Vision Transformer research, this thesis will explore a ViT-assisted metric related to 
image compression. This metric can provide additional insights to GAN output quality 
and the latent space (contextual) preservation of a variety of input images.


\NOTE{Thus, this work can be viewed as a stepping stone towards an end-to-end Transformer-based
image compression and regeneration framework.}


\section{Brief History}

\subsection{Attention and Language Models}

"Attention Is All You Need" is a seminal research publication
by a team of Google researchers, which kickstarted the Transformer revolution in Deep Learning in 2017.
It proposes a novel architecture, which models long-range dependencies in
sequential (text) data, by arranging a set of self-attention layers. 

A self-attention layer is what the model uses to focus on different elements of the 
input sequence simultaneously. For example, it can be ysed to compute the 
distance (relationship) between every word in a given sentence. \citep{Attention}


Examples of implementations of text-based Transformers are BERT by Google and GPT-3 by OpenAI.
BERT, among other things, as of 2021 processes and autofills every single English-based 
Google user search query. \citep{bert}
GPT-3, on the other hand, revolutionized text generation in 2020, demonstrating the ability 
to generate extremely cohesive textual output.

Most Transformers are used in language modeling and Natural Language Processing (NLP).
Thus, they are often benchmarked against Recurrent Neural Networks (RNNs, and specifically 
Long Short-Term Memory, LSTM architecture). LSTMs rely on hidden states to pass information 
along sequentially during the encoding and decoding process for each word token. 
However, they typically fall short learning long-range dependencies.


\subsection{Attention in Vision Tasks}

The attention mechanism is capable of focusing on objects found anywhere on an input image.
It operates within a single network layer compared to Convolutional Neural Networks (CNNs),
where the variable size convolution kernels scan across the different layers of the architecture.
\citep{dosovitskiy2020vit}

Tokenization happens at the pixel level, i.e. each pixel attends to each other pixel in the grid. 
This becomes computationally intensive, on the order of $(n^2)^2$, where $n$ denotes width of a square image. 
To resolve this, the input image is broken down into square blocks of equal size, referred to as image patches.
Then, each image patch is unrolled into a one-dimensional sequence $(n x 1)$ and indexed with a positional 
embedding in a table for future reference and retrieval purposes. The embeddings enter the Transformer and 
finally, a feed forward classifier, in the form of a Multilayer Perceptron (MLP) 
makes the classification prediction, yielding a probability distribution.\citep{dosovitskiy2020vit}


Transformer, in a way, is a generalization of a feed forward network, but 
instead of fixed connections weights in an MLP, each connection weight (i.e. attention) 
is computed ad hoc. That makes the Transformer, unlike the MLP, permutation invariant. 
That is, it would not know where certain information is coming from, unless there are additional 
learnable sequential positional embeddings, i.e. index the image patches.


\section{Principles of Operation}

Continuing from the previous section, a good way to think of a ViT is as a 
generalization of an MLP, which itself is a generalization of a CNN. 
The ViT happens to learn very similarly to a CNN, which represents the latent space
as filters carrying principal components.

In principle, CNNs have good inductive priors and can learn any function. 
However, they promote locality, i.e. nearing pixels are probability-wise considered most important.
This may easily not be desired, especially in the key applications of object detection and, in the 
future, image compression.


The encoding process indexes embeddings. For instance, certain key words in a sentence or 
objects in image blocks are mapped in a reference lookup table. 
The Decoder outputs Keys at each step. These vectors represent hidden states,
which are being passed on into each next iteration of the Transformer. 
The last layer, expectedly, uses a Softmax architecture to normalize and map the potential output classes 
to a probability distribution. 


\textbf{Multi-Head Attention}


Sets of parallel attention layers at each token are called multi-head attention 
(to vary what to pay attention to: e.g. at verbs in the sentence or different objects 
in an image). The multi-head attention is composed of Key-Value pairs coming from 
the encoding part of the source sentence or image (i.e. the input embedding) and 
Queries from the output embedding (i.e. encoding part of target sentence or image).
Now that we explained relevant transformer components, we can see how it applies to 
2D signals, i.e. image matrices for classification purposes in image recognition.


\section{Mathematical Formulation}

So in its full formulation, Attention is a function of queries, keys and values 
vectors labeled capital $(Q,K,V)$. 
It equals the dot product $(QK^T)$ of keys and queries respectively, softmaxed over 
the square root of dimensions and multiplied by Values.
So:
Values - are what is most interesting in the source (sentence or image), 
e.g. attributes or features (like keyword adjectives or perhaps structural 
features in an image).
Keys - index (or address) those values (name, type, weight). Each key has an 
associated value. Queries - are built by the encoder of the target sentece or 
image and prompt the network to find information. 

\begin{center}
	$ Attention(Q, K, V) = softmax(\dfrac{QK^T}{\sqrt{d_k}} V ) $
	\end{center}

The overall dynamic: a Query is pegged against a Key to locate a certain Value.	

The softmax is basically a normalized exponential function: sequence of Variables is 
mapped into exponentials and divided by the sum of all the exponentials. Thus, the 
large numbers become almost ones and small numbers near zeros, like the maximum 
function, but this one is differentiable. 

\begin{center}
	$ \displaystyle\sigma(Z_i) = \dfrac{e^{z_j}}{\sum_{j=1}^K e^{z_j}}$ for $i=1,...,K$ $and z=(z_1,...,z_K) \in \mathbb{R}^K $
	\end{center}

So a softmax of an inner product of each key with query vector normalizes to a 
probability distribution over all Values (very similar to using a softmax in the last 
layer of a NN over all the labels to yield the top classification pick). 


To understand vector proximity between embeddings, e.g. similarity in objects for images, or words in sentences.

\begin{figure}[H]
	\begin{center}
	\includegraphics[width=0.5\textwidth]{media/dot_real.png}
	\end{center}
	\caption[Vector Representation]{Vector representation on a unit circle.}
	\end{figure}

The dot product of Keys and Queries yields an angle between both vectors (e.g. $u_i$ and $u_j$ in Figure 1.2 above) to measure 
how similarly aligned they are. In high dimension, most vectors would be orthonormal 
and $cos(90)=0$. But if Key and Query align, they'd have a large dot product. 
Each key in space has an associated value (the pair). The Query vector is computed with 
each key and softmaxed to select one Key with the highest dot product. With softmax, 
a certain Key will stand out (in magnitude) vs the rest.

\begin{figure}[H]
	\begin{center}
	\includegraphics[width=0.3\textwidth]{media/dot2_real.png}
	\end{center}
	\caption[Key/Query Vector Proximity]{Vector proximity shows the closest Key vector to a given Query vector.}
	\end{figure}

So the formulation of each proximity is the dot product of vectors K with Q: $ \displaystyle sum(\langle K_z \vert Q \rangle) $

%\newpage

\section{Implementations}
\subsection{Open Source Pre-trained Models}

Open Source
On GitHub.
Original one by Google in tensorflow.
Hugging Face in PyTorch.
For the purpose of this thesis, we will use a PyTorch implementation trained on ImageNet-21k
and fine tuned on ImageNet-1k.

\subsection{Closed Source}

OpenAI (DALL.E 2) released in April 2022, trained on 250M image-text pairs 
to be able to generate images from textual description. \citep{Dalle2}
Not much information, not open source like google's ViT or BERT.

\section{Computational Constraints}

Time, memory, cost

Text transformers perform extremely well on orders of magnitude smaller size training examples.
GPT-3 trained on 45TB of text data (Wikipedia included), has 175B parameters and 96 attention layers. \citep{GPT3}
\$4.6M using a Nvidia advanced datacenter GPU grade cloud cluster.

Several orders of magnitude more for image data.

Need a ViT on all internet, to cost \$100M in training alone. 
To train a ViT on the whole TACC Frontera at 20k teraflops (top10) or 
Stampede at 10k tflops (top25), it would take respectively about a minute and 2 minutes.
tpu v3 is 420 teraflops * 2500 = 1M Tflops












%============================================================================%
%
%	CHAPTER 2
%
%============================================================================%
\cleardoublepage
\chapter{Specific Tools for a Thesis in Computer Science}

This chapter covers more specific examples, which are useful in the field of computer sciences.

\newpage
\section{"An Image is Worth 16x16 Words"}

\subsection{Transformers for Image Recognition at Scale}

Now that we explained relevant transformer components, we can see how it applies to 2D signals, i.e. image matrices for classification purposes in image recognition.
The Vision Transformer or ViT, was very recently published for ICLR 2021 by a google team for a ImageNet trained transformer.
Tokenization happens at pixel level, so each pixel would have to attend to  each other pixel in the grid, which becomes too heavy to compute, on the order of $(250^2)^2$. 
To resolve that, the image is broken down into blocks of equal size, a 16x16 subset of the image called image patches. Then, unroll each image patch into a sequence (256x1), and index it with a positional embedding in a table. All of that is then fed into a standard Transformer, like from Attention is all you need. Finally, a feed forward classifier (MLP) makes the classification prediction, voila image recognition.
The total number of parameters is on the order of 100M.

"image is worth 16x16 words" completely discards the notion of convolutions. Compared to a convolutional counterpart (the ResNet), the ViT cost 75% less to train and beats accuracy on ImageNet by 1%. ViT uses approximately 2 − 4× less compute to attain the same performance (averaged over 5 datasets). Unlike variable size convolution kernels accross layers, the block size in the image transformer is able to pay attention within a single layer to anywhere on the image. 
CNNs have good inductive priors and can learn any function. However, this promotes locality (i.e. nearing pixels are probably most important).
One way to think of a ViT is a generalization of an MLP which itself is a generalization of a CNN. The ViT also learns very similarly to a CNN (e.g. filters with principal components).
Transformer, in a way, is a generalization of a feed forward network, but instead of fixed connections weights in an MLP, each connection weight (i.e. attention) is computed on the fly. That makes the Transformer, unlike the MLP, permutation invariant. In that it wouldn't know where information is coming from unless there are additional learnable sequential positional embeddings, i.e. number down the image patches.

% \begin{figure}[H]
% \begin{algorithm}[H]
% \KwData{A as Array to sort,}
% \KwResult{A as sorted Array} 

% int n $\leftarrow$ A.size \tcp*[l]{cache the initial size of A}


% \Repeat{n>1}{
% 	int newn $\leftarrow$ 1
% 	\For{int i=0; i<n-1; i++}
% 	{
% 		\If{A[i] > A[i+1]}
% 		{
% 			A.swap(i, +1);
% 		}
% 	}
% 	n  $\leftarrow$ newn
% }

% return A
% \caption{bubbleSort(Array A)}
% \end{algorithm}
%\caption[TBC]{Figure caption.v.}
% \end{figure}


\newpage
\section{"Towards End-to-End Image Compression with Transformers"}


% \begin{table}[H]
% \begin{center}
% \begin{tabular}{|c|c|c|c|c|c|c|}
% \hline
% result XY & \multicolumn{6}{c|}{Rater A}\\\hline
%  \multirow{6}{*}{Rater B}& & \textbf{9} & \textbf{0} & \textbf{1} & \textbf{2} & \textbf{c(a) }\\\cline{2-7}
%  & \textbf{9}& 27	&0&	0	&0 & 0.5510204082\\\cline{2-7}
% &\textbf{0}	&0	&6	&0	&0&0.1224489796\\\cline{2-7}
% &\textbf{1}	&0	&1	&7	&0&0.1632653061\\\cline{2-7}
% &\textbf{2}	&0	&1	&7	&0&0.1632653061\\\cline{2-7}
% &\textbf{c(r)} & 0.5510204082 & 0.1632653061 & 0.2857142857 & 0& n=49\\\hline

% \end{tabular}
% \caption[Example data for mathematical equations]{A set of example data for further use in a mathematical equation.  }
% \end{center}
% \end{table}

\begin{center}
$Pr(e) = \displaystyle\sum_{i=1}^{m} c(r)_i \times c(a)_i $
\end{center}


\section{"TransGAN"}
\subsection{"Two Pure Transformers Can Make One Strong GAN"}

Sometimes you want to include text which include characters that could trigger commands. At this point it useful to wrap them into the \NOTE{verbatim} environment. The text is uninterpreted and listet as is.

You can also include existing code and highlight it's syntax, using the \NOTE{lstlisting} package. Look at the declarations.tex file for listings settings. The following example highlights xml syntax by given keywords.


\section{"First Principles of Deep Learning and Compression"}

Deterministic, Probabilistic
GANs
JPEG/MPEG

\section{Commonly Used Metrics for Image Quality}

SSIM, MSE, PSNR, BRISQUE

%============================================================================%
%
%	BIBLIOGRAPHY
%
%============================================================================%

\cleardoublepage
\phantomsection %hyperref package support
\addcontentsline{toc}{chapter}{Bibliography} % add entry to table of contents
\pagestyle{plain}


%{\textbf{\LARGE{Bibliography}}}\\	%headline
%\nocitep{*} % Show all Bib-entries (DEBUG)

\bibliographystyle{abbrvnat}
% \bibliography{bib/algorithm.bib,bib/resDes.bib} % for a better structure you can split your bib items into seperate files
\bibliography{bib/resDes.bib}


\end{document}