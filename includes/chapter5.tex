\section{Key Contributions (experimentation with Vision Transformers, nascent field)}


The package \NOTE{algorithm2e} allows you write nearly all imaginable code structures as pseudo code in LaTex. A referece of commands can be found here: \url{http://www.cs.toronto.edu/~frank/Useful/algorithm2e.pdf}. Do not forget to wrap the pseudo code into a figure, thus it will be included into list of figures.\\

The following example demonstrates the package with a bubble sort algorithm. Note, that the directive \NOTE{repeat - until} is the same as the \NOTE{do-while} directive.


\newpage
\section{Takeaways}

Writing mathematical equations can be a very important part of your work. This applies especially, if you want to analyze and evaluate your gathered data. The following example shows a calculation of Kohen's Cappa, a measurement which represents the inter coder reliability in coding qualitative data.\\

Let's assume the following results of coding a text by two different persons:


Continuing from this basis, the calculation can be proceeded with Cohen's Kappa defined as


\section{Acknowledgments (LIVE lab, TACC)}

Sometimes you want to include text which include characters that could trigger commands. At this point it useful to wrap them into the \NOTE{verbatim} environment. The text is uninterpreted and listet as is.


\section{Closing Remarks}

Sometimes you want to include text which include characters that could trigger commands. At this point it useful to wrap them into the \NOTE{verbatim} environment. The text is uninterpreted and listet as is.
