\section{Generative Image Compression and Generation}

(vineeth)

\subsection{Architecture}

SGD optimizer for GAN

\subsection{Sample Images Used}

\begin{figure}[H]
	\begin{center}
	\includegraphics[width=0.7\textwidth]{media/input_images_labeled.png}
	\end{center}
	\caption[Input Images]{Input Images used in this project (512x512):
    \newline
    \hspace{10pt} "Bevo": The University of Texas mascot, a famous longhorn bull.\newline
    \hspace{10pt} "Kliment": A face portrait of the author. \newline
    \hspace{10pt} "Logo": The Texas Longhorns logo.\newline
    \hspace{10pt} "Tower": The University of Texas Tower, the Main Building on campus.}
	\end{figure}

        % \begin{figure}[H]
% \begin{algorithm}[H]
% \KwData{A as Array to sort,}
% \KwResult{A as sorted Array} 

% int n $\leftarrow$ A.size \tcp*[l]{cache the initial size of A}


% \Repeat{n>1}{
% 	int newn $\leftarrow$ 1
% 	\For{int i=0; i<n-1; i++}
% 	{
% 		\If{A[i] > A[i+1]}
% 		{
% 			A.swap(i, +1);
% 		}
% 	}
% 	n  $\leftarrow$ newn
% }

% return A
% \caption{Neural Image Compression}
% \end{algorithm}
% \caption[TBC]{Figure caption caption}
% \end{figure}

\subsection{Latent space vector representation during compression}

$nx1$ vector, where $n$ corresponds to the height or width (in pixels)
of a square input image. In the case of all input images used, the latent vector
is of size $512x1$, since the input images are of size $512x512$.

hard to decipher but here is a zoomed in visual of the first
This is what the GAN architecture compresses the full image to (1.54kb vs original size 409kb).
The GAN then rebuils to 242kb.


\begin{figure}[H]
	\begin{center}
	\includegraphics[width=0.5\textwidth]{media/latent.png}
	\end{center}
	\caption[Latent Space Representation]{Visual representation of the first 10 pixels in the most compressed version of the original image.}
	\end{figure}

% \newpage
\section{Output and Visual Inspection}
The generative process from the GAN was designed to output an image at a specified epoch.

\subsection{Training Process}
Below is a demonstration of the GAN learning process at every 250 epochs:
\begin{figure}[H]
	\begin{center}
	\includegraphics[width=0.8\textwidth]{media/ganLearn.png}
	\end{center}
	\caption[GAN Training Process]{The GAN learns to compress and generate the Texas Longhorns logo.}
	\end{figure}

Due to the cutting edge nature of the technologies used, a natural performance asymptote
was observed. The GAN was able to reconstruct certain input images better than others.
After a certain iteration, as usual, the GAN was unable to further learn how to compress,
represent, and regenerate some input images. Typically, once a GAN reaches this stage,
it learns from random noise and generation performance decreases.

\subsection{Generated Results}

Figure 3.4 below shows the resulting images from the neural compression GAN.

\begin{figure}[H]
	\begin{center}
	\includegraphics[width=0.8\textwidth]{media/gan_compress2.png}
	\end{center}
	\caption[Neural Compression and Generation]{After sufficient training, the GAN outputs a regenerated
    version of the original image from a latent space vector representation.}
	\end{figure}

The GAN was able to respectably regenerate "Kliment" and "Logo", especially if the resolution
were to be lowered (e.g. 32x32, such as the CIFAR-10 dataset).

However, the GAN was unable to perform as well for "Bevo" and "Tower".
It learned random noise and generation performance decreased.

\section{ViT-Scores}

The Vision Transformer-Assisted ViT Score is an original development from this thesis. 

It is an attempt to measure the quality of a generated image after neural compression.

The ViT-score is in the open interval $(0,1)$ with $0$ being poor and extremely dissimilar from
the original and $1$ being excellent and fully similar to original.

Mathematically, the endpoint values of the interval are unattainable by probabilistic models
such as the GAN.

\subsection{Mathematical Formulation}

The following is a mathematical representation explained in further detail.

\begin{center}
    $ViT_{score} = \displaystyle\dfrac{argmax_{A'\subset A,\lvert A' \rvert = k } \sum_{a \in A'} {a} }{k} $
\end{center}


where $\sum_{a \in A'} {a}  = \lbrace{m \in I_{input}}\rbrace \cap \lbrace{n \in I_{generated}}\rbrace$


and $m$ are the top-$K$ labels in the input image $I_{input}$

and $n$ are the top-$K$ labels in the generated image $I_{generated}$.

\vspace{12pt}

This overly elaborate mathematical notation is an attempt at describing:


\NOTE{
"From the full set of trained ViT labels, we find the top-$K$ number of intersecting labels between the original and generated images. Then, we divide that by $K$"
}


For example, of the top-100 labels found in the original image, 
identify the set of labels also found in the generated image. 
Then, divide that number of intersecting
labels by the total number of 100 labels.


\subsection{ViT-Scores from Resulting Images}


Following Figure 3.7, the ViT-scores for the GAN generated images after neural compression
are as follows:

\begin{center}
	\includegraphics[width=0.6\textwidth]{media/gan_compress2.png}
\end{center}

\begin{table}[H]
\begin{center}
\begin{tabular}{|c|c|}
\hline
\textbf{Bevo}	& 0.14\\
\textbf{Kliment}	& 0.54\\
\textbf{Logo}	& 0.29\\
\textbf{Tower} & 0.03\\\hline

\end{tabular}
\caption[ViT-Scores of Generated Images]{ViT-Scores demonstrate a somewhat expected quality assessment.}
\end{center}
\end{table}



\textbf{"Kliment"} leads with a ViT-score of $0.54$, which is understandable as the GAN 
generated a face (although smudgy) and was rather able to recreate the scenery structurally.

\textbf{"Logo"} generation seems structurally excellent and the ViT-score is $0.29$, which
is considered a good score for this particular GAN architecture and training.

\textbf{"Bevo"} barely preserves the original shape at ViT-score of $0.14$, while
\textbf{"Tower"} is incomprehensible and barely resembles the original at ViT-score of $0.03$.


Overall, the ViT-score does a good job of measuring image quality.


\section{Established IQA Metrics}

\subsection{"Kliment"}

\begin{figure}[H]
	\begin{center}
	\includegraphics[width=1\textwidth]{media/kimbo_metrics.png}
	\end{center}
	\caption["Kliment" Established Metrics]{After sufficient training, the GAN outputs a regenerated
    version of the original image from a latent space vector representation.}
	\end{figure}

\subsection{"Logo"}

\begin{figure}[H]
	\begin{center}
	\includegraphics[width=1\textwidth]{media/logo_metrics.png}
	\end{center}
	\caption["Logo" Established Metrics]{After sufficient training, the GAN outputs a regenerated
    version of the original image from a latent space vector representation.}
	\end{figure}


\subsection{"Bevo"}

\begin{figure}[H]
        \begin{center}
        \includegraphics[width=1\textwidth]{media/bevo_metrics.png}
        \end{center}
        \caption["Bevo" Established Metrics]{After sufficient training, the GAN outputs a regenerated
        version of the original image from a latent space vector representation.}
        \end{figure}

\subsection{"Tower"}

\begin{figure}[H]
        \begin{center}
        \includegraphics[width=1\textwidth]{media/tower_metrics.png}
        \end{center}
        \caption["Tower" Established Metrics]{After sufficient training, the GAN outputs a regenerated
        version of the original image from a latent space vector representation.}
        \end{figure}

\subsection{"BRISQUE"}

Blind/Referenceless Image Spatial Quality Evaluator (BRISQUE)
where approaching 0 is a good score and approaching 100 is a bad score, the BRISQUE referenceless 
image quality methodology

This score could be interpreted as the image being more photorealistic than not. 
In terms of quality, this compares to a camera captured image with quality corruption 
caused by blurs or graininess. An image with no distortions often has a score below 5.

\begin{table}[H]
    \begin{center}
    \begin{tabular}{|c|c|c|}
    \hline
      & Original & Generated \\ [0.5ex] 
    \hline\hline
    \textbf{Bevo}	& 32.9214 & 39.5535\\
    \textbf{Kliment}	& -8.3593 & 44.3570\\
    \textbf{Logo}	& 102.9010 & 97.1844\\
    \textbf{Tower} & 14.5973 & 52.8363\\\hline
    
    \end{tabular}
    \caption[BRISQUE]{BRISQUE Scores of original and generated images.}
    \end{center}
    \end{table}

Expectedly, the BRISQUE values for the generated images are always higher than
their original counterparts. "Logo" is not a photorealistic image to begin with, 
so it is understandable that the BRISQUE value is high at $102.9$.
None of the generated images would pass BRISQUE as photorealistic and free of 
distortions.  

Loss functions as well (MSE loss was used in GAN)


\section{GAN-Related Quantitative Metrics}


(FID score, inception score (IS) )
Can be loss functions as well (MSE was used)
FID is Frechet Inception Distance. 0 if there is no difference between the images.
81105.162 for logo and logo\_GAN.
331171.556 for tower and towerGAN
549089.491 for kimbo and kimboGAN
1241999.901 for bevo and bevoGAN


\section{Summary of Results}

\begin{center}
	\includegraphics[width=1\textwidth]{media/gan_compress2.png}
\end{center}

\begin{table}[H]
    \begin{center}
    \begin{tabular}{|c|c|c|c|c|c|c|}
    \hline
    \textbf{Image}	& \textbf{ViT-Score} & \textbf{SSIM} & \textbf{MSE} & \textbf{PSNR} & \textbf{FID} & \textbf{BRISQUE} \\
    \hline
    \textbf{Bevo}	 & 0.14 & 0.26 & 3,479.34  &  12.68  &  1,241,999.901  &  39.5535\\
    \textbf{Kliment} & 0.54 & 0.56 & 1,759.28 & 15.64  &  549,089.491 & 44.3570\\
    \textbf{Logo}	 & 0.29 & 0.91 & 393.69 & 22.15 &  81,105.162 & 97.1844\\
    \textbf{Tower}   & 0.03 & 0.53 & 1,315.75 & 16.90 &  331,171.556 & 52.8363\\\hline
    
    \end{tabular}
    \caption[Summary of Results]{ViT-Scores demonstrate a somewhat expected quality assessment.}
    \end{center}
    \end{table}