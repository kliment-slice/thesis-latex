\section{Architecture}

\subsection{Main Components}

The main architectural components of this project are a Generative Adversarial Network (GAN) and a Vision Transformer (ViT).
An input image is compressed into a latent vector form, which is then used to generate a synthetic version of the original image.
The GAN iteratively improves its ability to generate a realistic image, which resembles the input (refer to Section 2.3 for an overview of GANs).
A pre-trained "PGAN" was used, further described in Section 3.1.3.

A pre-trained Vision Transformer (ViT) was used for object detection in both the input and reconstructed images. 
A custom method then identifies the number of matching labels between the images. Finally, a ViT-Score is computed. 

\begin{figure}[H]
	\begin{center}
	\includegraphics[width=1\textwidth]{media/main_arch2.png}
	\end{center}
	\caption[Model Architecture]{The full ViT-Scoring engine. A GAN compresses, then reconstructs
    an input image from its latent vector. Then, a Vision Transfomer (ViT) detects objects in the input and generated images.
    Finally, the ViT-Score is computed from the number of matching labels.}
	\end{figure}

\subsection{Image Compression Diagram}


Stochastic Gradient Descent (SGD) optimizer was used for the GAN to find the most optimal latent vector representation of the input image.
Mean Squared Error (MSE) was used as a Loss function to improve the GAN performance.

Similar to established image compression methodologies, the compression and reconstruction process 
is divided into a Compressor and Decompressor parts.

\begin{figure}[H]
	\begin{center}
	\includegraphics[width=0.6\textwidth]{media/main_compress.png}
	\end{center}
	\caption[Compression Architecture]{The Compressor-Decompressor uses the GAN Generator from Figure 3.1 above 
    to generate the output image using most optimal latent vector representation.}
	\end{figure}

As showing in Figure 3.2 above, the Compressor is composed of the GAN Generator and an input image to find and return the latent vector using
Stochastic Gradient Descent (SGD).

The Decompressor then uses the GAN Generator and latent vector from the Compressor to produce the final reconstructed image.


Following the Decompressor generating an output image, the pre-trained ViT then detects objects from a 
dictionary of labels it has been trained on. The ViT outputs a probability distribution, i.e. a probability for each label summing to unity (Softmax, refer to Section 1.4.2). 
The final output of the script developed for this project is the ViT-Score, and other relevant image quality metrics (e.g. SSIM, MSE, etc)
to evaluate the generated output image.

A PyTorch implementation of the ViT was used, trained on ImageNet-21k and fine tuned on ImageNet-1k (refer to Section 1.5.1).

Python and Python libraries were used throughout the implementation of the architectures described above (see Appendix A for details).

\subsection{The GAN}

A standard Generator-Discriminator GAN architecture, called a "PGAN" (Progressive Growing of GAN), was used from a pre-trained model.
The model was trained on three major datasets: "celebaHQ", "fashionGen", and "DTD". 
"CelebaHQ" is a dataset of celebrity faces. "FashionGen" is a set of fashion objects such as clothing and accessory items.
"DTD" (Describable Textures Dataset) is a collection of textures such as checkered patterns, foods, and animal fur.
\citep{PytorchGANZoo}

\begin{figure}[H]
	\begin{center}
	\includegraphics[width=0.7\textwidth]{media/pgan.png}
	\end{center}
	\caption[The pre-trained "PGAN"]{The pre-trained "PGAN" by Facebook-Research, implemented in PyTorch. 
    It is trained on three diverse datasets of objects.\citep{PytorchGANZoo}}
	\end{figure}

\NOTE{The presumption is that since a "PGAN" is trained on these three diverse datasets, it should, in principle, be able to compress 
and reconstruct any input image. Its output however, may vary significantly, depending on the complexity of the input.}

\subsection{Latent space vector representation}

A square input image entering the compression engine is compressed to a $nx1$ vector, where $n$ corresponds to the height or width (in pixels)
of the image. In the case of all input images used in this project, the latent vector
is of size $512x1$, since the input images are of size $512x512$.

The exact meaning of the vector is hard to decipher. Figure 3.4 below shows a sample of the compressed image vector representation:

\begin{figure}[H]
	\begin{center}
	\includegraphics[width=0.4\textwidth]{media/latent.png}
	\end{center}
	\caption[Latent Space Representation]{Visual representation of the first 10 pixels of the most compressed form of the original image.}
	\end{figure}

The figure above demonstrates what the GAN architecture compresses the full image to. In this case, an original image 
of size $409$ kilobytes was compressed to a $1.54$ kilobyte image vector. 

The resulting Compression Ratio (CR) is $266:1$.
Such a Compression Ratio is an order of magnitude higher than deterministic compression algorithms would ever be able to achieve. 
\citep{Principles}

The GAN then proceeds to rebuild the image to 242kb. The final Compression Ratio is $1.69$.
% \newpage
\subsection{Sample Input Images Used}

Four images of size 512x512, in PNG format are used in this study.
These images represent distinct types of image to be compressed, containing 
diverse features, in order to challenge the model architecture.

For example, one image ("Logo") contains a white background, while another ("Bevo") a black background.
The two completely dissimilar images, "Logo" and "Bevo", also happen to contain similar shapes, representing a graphic and a living
Texas longhorn bull. "Bevo" contains an animal head and face.
The image "Kliment", a portait of the author, contains a human head and face with trees and clouds in the background.

The image "Tower" is a building, the University of Texas at Austin Main Tower, containing
trees in the foreground, but clouds in the background.

\NOTE{This set of four images was carefully chosen to represent similarities and dissimilarities in order to 
test the robustness of the ViT-Score and capacity of the Vision Transformer itself.}

\begin{figure}[H]
	\begin{center}
	\includegraphics[width=0.5\textwidth]{media/input_images_labeled.png}
	\end{center}
	\caption[Input Images]{Input Images used in this project (512x512):

    \hspace{2cm} "Bevo": The University of Texas mascot, a famous longhorn bull.

    \hspace{2cm} "Kliment": A face portrait of the author. 

    \hspace{2cm} "Logo": The Texas Longhorns logo.
    
    \hspace{2cm} "Tower": The University of Texas Tower, the Main Building on campus.}
	\end{figure}


\section{Outputs and Visual Inspection}
The generative process from the GAN was designed to output an image at a specified epoch.
Along with manual stopping, criteria for early stopping was set as well. 

Moreover, learning rate reduction on plateau was used. This technique reduces the learning rate
hyperparameter as the loss stops reducing.

The generated images are output for the user to monitor the GAN learning process.

\subsection{Training Process}
Below is a demonstration of the GAN training to compress and reconstruct "Logo". 

Samples are collected at every 250 epochs:
\begin{figure}[H]
	\begin{center}
	\includegraphics[width=0.8\textwidth]{media/ganLearn.png}
	\end{center}
	\caption[GAN Training Process]{The GAN learns to compress and generate the Texas Longhorns logo.
    It starts from a blob, then 3500 epochs later, it is able to produce a completely recognizable version of "Logo". }
	\end{figure}

Due to the novel nature of the technologies used, a natural performance asymptote
was observed. The "PGAN" itself is trained on diverse data, yet this data is not able 
to capture the complexity of all potential input images. 
Thus, the GAN was able to reconstruct certain input images better than others.
After a certain iteration, as usual, the GAN was unable to further learn how to compress,
represent, and regenerate some input images. 

Typically, once a GAN reaches this stage,
it learns from random noise and generation performance decreases.
Losses start increasing and output image quality degrades. 

\subsection{Generated Results}

Figure 3.7 below shows the resulting images from neurally compressing and reconstructing
all four sample input images.

\begin{figure}[H]
	\begin{center}
	\includegraphics[width=0.8\textwidth]{media/gan_compress2.png}
	\end{center}
	\caption[Neural Compression and Generation]{After sufficient training, the "PGAN" outputs a regenerated
    version of the original image from its latent vector representation. All four reconstructions
    resemble their inputs, to a varying degree of success.}
	\end{figure}

The GAN was able to respectably regenerate "Kliment" and "Logo".
"Logo" is completely recognizable as the Texas Longhorns logo, especially if the resolution
is to be lowered (e.g. 32x32). In "Kliment", the model was successful at reconstructing a 
human head and the background features. 

Nevertheless, the GAN was unable to perform as well for "Bevo" and "Tower".
It learned random noise and generation performance decreased.
Though "Bevo" does resemble a living Longhorn bull when viewed next to its original,
a standalone image would be confusing to a viewer, as well as a Deep Learning-based object detector. 
"Tower" is clearly the worst output, since the generated image only loosely matches a blurry 
version of the original shape.

\subsection{Compression Ratios (CR)}

Compression Ratios (CR) are calculated as:
\begin{center}
	$ \displaystyle CR = \dfrac{original\_size}{compressed\_size}$
	\end{center}

However, it is important to note that the CR can technically be computed as the ratio of 
the original image size to the latent vector size.
Since the latent vector compresses to about $1.5$ kilobytes , 
the resulting compression ratio would always be around on the order of $100:1$ to $1000:1$.

This latent vector CR does not necessarily help with evaluating a transmitted image. 
For the purpose of this section, the size of the GAN reconstructed image is used as a more
conservative measurement of compression. 

\vspace{2mm}

\NOTE{This conservative measure of performance would also demonstrate 
model superiority over established compression techniques.}

\vspace{1mm}

\begin{table}[H]
\begin{center}
\begin{tabular}{|c|c|}
\hline
\textbf{Bevo}	& 0.75 \textit{(failed to compress)}\\
\textbf{Kliment}	& 1.69\\
\textbf{Logo}	& 0.12 \textit{(failed to compress)}\\
\textbf{Tower} & 1.85\\\hline

\end{tabular}
\caption[Results: Compression Ratios]{Compression ratios of generated images.}
\end{center}
\end{table}

The GAN failed to compress two of the images, "Bevo" and "Logo". 
Understandably, "Logo" is actually a vector graphic with a clear background, with an input file size of only 13.4 KB.
The GAN had to create the white background, which it inefficiently stores as information, costing extra 
bandwidth and disk space.
"Kliment" and "Tower" were compressed to half their original sizes. 
"Kliment" is a suitable candidate to demonstrate the capacity of this generative model.
The image has a high Compression Ratio (CR), while also maintaining a high ViT-Score.
After all, the most popular application and training set of most GANs, included in this pre-trained model (celebaHQ dataset),
is indeed human faces.


\section{ViT-Scores}

The Vision Transformer-Assisted, ViT-Score is an original development from this thesis. 

It is an attempt to measure the quality of a generated image after neural compression.

The ViT-score is in the open interval $(0,1)$ with $0$ being poor and extremely dissimilar from
the original and $1$ being excellent and fully similar to original.

Mathematically, the endpoint values of this interval are unattainable by the ViT-Score. 
Probabilistic models such as this GAN, would generate enough similarities, though never an exact match.

\NOTE{The ViT-Score can be considered a contextual measure. It captures a higher abstraction of 
pixel arrangement based on objects detected under a certain probability. Such measures can be insightful, 
especially when evaluating generative models.}

\subsection{Mathematical Formulation}

The following is a mathematical representation later explained in further detail.

\vspace{1mm}


\begin{center}
    $ViT_{score} = \displaystyle\dfrac{argmax_{A'\subset A,\lvert A' \rvert = k } \sum_{a \in A'} {a} }{k} $
\end{center}

\vspace{1mm}

\begin{center}
where $\sum_{a \in A'} {a}  = \lbrace{m \in I_{input}}\rbrace \cap \lbrace{n \in I_{generated}}\rbrace$
\end{center}

\vspace{2mm}

and $m$ are the top-$K$ labels in the input image $I_{input}$

and $n$ are the top-$K$ labels in the generated image $I_{generated}$.

\vspace{5pt}

This overly elaborate mathematical notation is an attempt at describing:

\vspace{1mm}

\NOTE{
"From the full set of trained ViT labels, we find the top-$K$ number of intersecting labels between the original and generated images. Then, we divide that by $K$"
}

\vspace{4mm}    

For example, of the top-100 labels found in the original image, 
identify the set of labels also found in the generated image. 
Then, divide that number of intersecting
labels by the total number of 100 labels.


\subsection{ViT-Scores from Resulting Images}


Following Figure 3.7, the ViT-scores for the GAN generated images after neural compression
are as follows:

\begin{center}
	\includegraphics[width=0.6\textwidth]{media/gan_compress2.png}

    \textbf{Resulting generated images.}
\end{center}

\newpage

\textbf{ViT-Scores}

\begin{table}[H]
\begin{center}
\begin{tabular}{|c|c|}
\hline
\textbf{Bevo}	& 0.14\\
\textbf{Kliment}	& 0.54\\
\textbf{Logo}	& 0.29\\
\textbf{Tower} & 0.03\\\hline

\end{tabular}
\caption[ViT-Scores of Generated Images]{ViT-Scores demonstrate a somewhat expected quality assessment.}
\end{center}
\end{table}



\textbf{"Kliment"} leads with a ViT-Score of $0.54$, which is understandable as the GAN 
generated a face (although smudgy) and was rather able to recreate the scenery structurally.

\textbf{"Logo"} generation seems structurally excellent and the ViT-Score is $0.29$, which
is considered a good score for this particular GAN architecture and training.

\textbf{"Bevo"} barely preserves the original shape at ViT-Score of $0.14$, while
\textbf{"Tower"} is incomprehensible and barely resembles the original at ViT-score of $0.03$.

\vspace{4mm}

\NOTE{Overall, the ViT-Score does a good job of measuring image quality.}


\section{Established IQA Metrics}

Refer to Section 2.5 "Metrics for Image Quality" for detailed definitions 
of each measurement technique.


\subsection{"Kliment"}

\begin{figure}[H]
	\begin{center}
	\includegraphics[width=0.8\textwidth]{media/kimbo_metrics.png}
	\end{center}
	\caption["Kliment" Established Metrics]{The image is structurally reconstructed well.
    Shortcomings are found in the facial features within the face.}
	\end{figure}


This image achieves a ViT-Score of $0.54$, SSIM of $0.56$, MSE of $1,759.28$, and a PSNR of $15.64$ with a Compression Ratio of $0.59$.
Major shortcomings have less to do with structure, and more with details and features.

\subsection{"Logo"}

\begin{figure}[H]
	\begin{center}
	\includegraphics[width=0.8\textwidth]{media/logo_metrics.png}
	\end{center}
	\caption["Logo" Established Metrics]{The GAN is able to reconstruct an almost 
    perfectly recognizable version of the original logo.}
	\end{figure}

The generated logo is extremely well identifiable.
This image achieves a ViT-Score of $0.29$, SSIM of $0.91$, MSE of $393.69$, and PSNR of $22.15$. However, it fails to compress
less than its original size. The GAN had to reconstruct the white background, which comes out off-white.

\subsection{"Bevo"}

\begin{figure}[H]
        \begin{center}
        \includegraphics[width=0.8\textwidth]{media/bevo_metrics.png}
        \end{center}
        \caption["Bevo" Established Metrics]{The GAN was unable to reconstruct details inside the longhorn, yet
        the structure of the image is decently rebuilt. One could possibly identify the animal from the generated image.}
        \end{figure}

"Bevo" achieves a ViT-Score of $0.14$, SSIM of $0.26$, MSE of $3,479.34$, and PSNR of $12.68$ while failing to compress at all. 

\subsection{"Tower"}

\begin{figure}[H]
        \begin{center}
        \includegraphics[width=0.8\textwidth]{media/tower_metrics.png}
        \end{center}
        \caption["Tower" Established Metrics]{The GAN completely failed at reconstructing a comprehensible image.}
        \end{figure}

This reconstruction is the lowest quality of all four images and we rightfully expect to have the lowest scores on all metrics.
Both quantitatively and qualitatively, the generated image is considered incomprehensible. 
It has the lowest ViT-Score at $0.03$, SSIM of $0.53$, MSE of $1,315.75$, and PSNR of $16.90$. The fact that it does compress is 
irrelevant, since the image lost structural, textural, and feature detail information.

\subsection{"BRISQUE"}

The Blind/Referenceless Image Spatial Quality Evaluator (BRISQUE) metric is also defined in Section 2.5.
BRISQUE measures distortions, with the score approaching 0 is a good score and approaching 100 is a bad score. 
This methodology could help evaluate an image as being more photorealistic than not. 
In terms of GAN output quality, this compares to a camera captured image with quality corruption 
caused by blurs or graininess.

\begin{table}[H]
    \begin{center}
    \begin{tabular}{|c|c|c|c|}
    \hline
      & Original & Generated & Delta\\ [0.5ex] 
    \hline\hline
    \textbf{Bevo}	& 32.9214 & 39.5535 & 6.6\\
    \textbf{Kliment}	& -8.3593 & 44.3570 & 52.7\\
    \textbf{Logo}	& 102.9010 & 97.1844 & 5.7\\
    \textbf{Tower} & 14.5973 & 52.8363 & 38.2\\\hline
    
    \end{tabular}
    \caption[BRISQUE]{BRISQUE Scores of original and generated images.}
    \end{center}
    \end{table}

Expectedly, the BRISQUE values for the generated images are always higher than
their original counterparts. That is, image quality always worsens after compression. 
"Logo" is not a photorealistic image to begin with, 
so it is understandable that the BRISQUE value is high at $102.9$.
None of the generated images would pass BRISQUE as photorealistic and free of 
distortions. Largest Deltas (changes from original to generated) are also found in 
the images with the highest Compression Ratios (CRs), "Kliment" and "Tower".


A MATLAB implementation of BRISQUE was used to compute these scores.


\subsection{GAN-Related Quantitative Metrics}

There are two prominent evaluation metrics for the performance of a Generative Adversarial Network (GAN).
Specifically, the Frechet Inception Distance (FID) and Inception Score (IS). Both measurements serve to 
evaluate the synthetic nature of the generated output. 

The FID score, defined in Section 2.5, is 0 if there is no difference between the two multidimensional 
Gaussian distributions compared. 

\vspace{5mm}

The \textbf{FID} scores are presented in Table 3.1 below:

\begin{table}[H]
    \begin{center}
    \begin{tabular}{|c|c|}
        \hline
    \textbf{Bevo}	& 1,241,999.901 \\
    \textbf{Kliment}	& 549,089.491\\
    \textbf{Logo}	& 81,105.162\\
    \textbf{Tower} & 331,171.556\\\hline
    
    \end{tabular}
    \caption[FID Score]{The Frechet Inception Distance (FID) score for all test images.}
    \end{center}
    \end{table}


The FID score can be used as a Loss functions as well, embedded within the architecture of the GAN.
However, it provides notoriously incosistent results, and is not as robut as MSE, which was used as 
the Loss function of choice for this project. None of the FID scores above provide much insight 
into the capacity of the GAN to generate a realistic image. 


\section{Summary of Results}

\begin{center}
	\includegraphics[width=0.8\textwidth]{media/gan_compress2.png}
\end{center}

\begin{table}[H]
    \begin{center}
    \begin{tabular}{|c|c|c|c|c|c|c|c|}
    \hline
    \textbf{Image}	& \textbf{ViT-Score} & \textbf{SSIM} & \textbf{MSE} & \textbf{PSNR} & \textbf{FID}  & \textbf{BRISQUE} & \textbf{CR}\\
    \hline
    \textbf{Bevo}	 & 0.14 & 0.26 & 3,479.34  &  12.68  &  1,241,999.901  &  39.5535 & 0.75 \\
    \textbf{Kliment} & 0.54 & 0.56 & 1,759.28 & 15.64  &  549,089.491 & 44.3570 & 1.69\\
    \textbf{Logo}	 & 0.29 & 0.91 & 393.69 & 22.15 &  81,105.162 & 97.1844 & 0.12 \\
    \textbf{Tower}   & 0.03 & 0.53 & 1,315.75 & 16.90 &  331,171.556 & 52.8363 & 1.85\\\hline
    
    \end{tabular}
    \caption[Summary of Results]{ViT-Scores provide an insightful quality assessment compared to established methods.}
    \end{center}
    \end{table}
